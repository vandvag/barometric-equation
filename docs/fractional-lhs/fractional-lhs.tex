\documentclass[12pt,letterpaper]{article}
\usepackage[utf8]{inputenc}
\usepackage[english]{babel}
\usepackage{amsmath}
\usepackage{amsthm}
\usepackage{amsfonts}
\usepackage{amssymb}
\usepackage{braket}
\usepackage{graphicx}
\usepackage[usenames]{color}

\theoremstyle{definition}
\newtheorem{ass}{Assumption}[subsection]
\newtheorem{theorem}[ass]{Theorem}
\newtheorem{lemma}[ass]{Lemma}
\newtheorem{prop}[ass]{Proposition}
\newtheorem{definition}[ass]{Definition}
\newtheorem{corollary}[ass]{Corollary}
\newtheorem{exer}[ass]{Exercise}
\newtheorem{ex}[ass]{Example}
\newtheorem{prob}[ass]{Problem}
\newtheorem{rem}[ass]{Remark}

\definecolor{darkolivegreen}{rgb}{0.333333, 0.419608, 0.1843140}

\usepackage[bookmarks=true,colorlinks=true,linkcolor=darkolivegreen,
citecolor=darkolivegreen,urlcolor=darkolivegreen]{hyperref}

\numberwithin{equation}{subsection}
\parindent  0.mm 
\allowdisplaybreaks

\begin{document}

The left-hand side of the equation is given by:

\begin{equation} \label{eq:system_matrix}
    M := \langle \phi_{ek} | [(A^*A)^{1/2} \, \phi_{em}] \rangle_{k,m=1,\dots,n}
\end{equation}
where:

\begin{equation}
    \phi_{ek}(x) =
    \begin{cases}
        \dfrac{x - (k-1)l}{l} \quad x \in [(k-1)l, kl] \\[10pt]
        \dfrac{(k+1)l - x}{l} \quad x \in [kl, (k+1)l]
    \end{cases}
\end{equation}
for $k \in \{1, \dots, n-1\}$ and for $k=n$:

\begin{equation}
    \phi_{en}(x) =
    \dfrac{x - (n-1)l}{l} \quad x \in [(n-1)l, nl] \quad x \in [(n-1)l, nl]
\end{equation}
Meanwhile, it holds true that:

\begin{multline}
    [(A^*A)^{1/2}\phi_{em}](y) = \\ \frac{1}{\pi l}\left\{ \ln{\left[\dfrac{\tan{\left(\dfrac{\pi[ml+y]}{4a}\right)}\tan{\left(\dfrac{\pi|(m-1)l-y|}{4a}\right)}}{\tan{\left(\dfrac{\pi|ml-y|}{4a}\right)}\tan{\left(\dfrac{\pi[(m-1)l+y]}{4a}\right)}}\right]} \right. \\
     \left. -  \ln{\left[\dfrac{\tan{\left(\dfrac{\pi[(m+1)l+y]}{4a}\right)}\tan{\left(\dfrac{\pi|ml-y|}{4a}\right)}}{\tan{\left(\dfrac{\pi|(m+1)l-y|}{4a}\right)}\tan{\left(\dfrac{\pi[ml+y]}{4a}\right)}}\right]}\right\}
\end{multline}
for every $m \in \{1,\dots,n-1\}$

\begin{equation}
    [(A^*A)^{1/2}\phi_{en}](y) = \frac{1}{\pi l} \ln{\left[\dfrac{\tan{\left(\dfrac{\pi[nl+y]}{4a}\right)}\tan{\left(\dfrac{\pi|(n-1)l-y|}{4a}\right)}}{\tan{\left(\dfrac{\pi|nl-y|}{4a}\right)}\tan{\left(\dfrac{\pi[(n-1)l+y]}{4a}\right)}}\right]}   
\end{equation}
Unpacking the equation (\ref{eq:system_matrix}):

\begin{equation}
    M_{km} = \int\limits_0^a \phi_{ek} (x)[(A^*A)^{1/2}\phi_{em}](x)dx
\end{equation}
We take into account that this integrand has a singularity at $x=kl$ with $k \in \{1, \dots, n-1\}$. We perform the following split. Also from now on we set $[(A^*A)^{1/2}\phi_{em}](x) = f_m(x)$:

\begin{equation*}
    M_{km} = \int\limits_0^{kl} \phi_{ek} (x)f_m(x)dx + \int\limits_{kl}^a \phi_{ek} (x)f_m(x)dx
\end{equation*}
We then introduce the term $|x - kl|^{\alpha}$, with $-1<\alpha<0$, in order to get rid of the singularity:

\begin{equation*}
    M_{km} = \int\limits_0^{kl} \phi_{ek} (x)f_m(x)|x - kl|^{-\alpha}|x - kl|^{\alpha}dx + \int\limits_{kl}^a \phi_{ek} (x)f_m(x)|x - kl|^{-\alpha}|x - kl|^{\alpha}dx
\end{equation*}
Setting $f_m(x) |x-kl|^{-\alpha} = g_m(x)$ we can write:

\begin{align*}
    M_{km} &= \int\limits_0^{kl} \phi_{ek} (x)g_m(x)(kl-x)^{\alpha}dx + \int\limits_{kl}^a \phi_{ek} (x)g_m(x)(x - kl)^{\alpha}dx\\
    &= \frac{kl}{2}\int\limits_{-1}^1 \phi_{ek}\left(\frac{kl}{2}\xi + \frac{kl}{2}\right)g_m\left(\frac{kl}{2}\xi + \frac{kl}{2}\right) \left(\frac{kl}{2}\right)^{\alpha}(1-\xi)^{\alpha}d\xi + \\
    &+ \frac{a-kl}{2} \int\limits_{-1}^1\phi_{ek}\left(\frac{a-kl}{2}\xi + \frac{a+kl}{2}\right)g_m\left(\frac{a-kl}{2}\xi + \frac{a+kl}{2}\right) \left(\frac{a-kl}{2}\right)^{\alpha}(1+\xi)^{\alpha}d\xi \\
    &= \left(\frac{kl}{2}\right)^{1+\alpha} \int\limits_{-1}^{1}\phi_{ek}\left(\frac{kl}{2}\xi + \frac{kl}{2}\right)g_m\left(\frac{kl}{2}\xi + \frac{kl}{2}\right)(1-\xi)^{\alpha}d\xi + \\
    &+ \left(\frac{a-kl}{2}\right)^{1 + \alpha} \int\limits_{-1}^1\phi_{ek}\left(\frac{a-kl}{2}\xi + \frac{a+kl}{2}\right)g_m\left(\frac{a-kl}{2}\xi + \frac{a+kl}{2}\right) (1+\xi)^{\alpha}d\xi \\
\end{align*}
We substitute:

\begin{equation}
    \tilde g_{1m}(\xi) = \phi_{ek}\left(\frac{kl}{2}\xi + \frac{kl}{2}\right)g_m\left(\frac{kl}{2}\xi + \frac{kl}{2}\right)
\end{equation}
\begin{equation}
    \tilde g_{2m}(\xi) = \phi_{ek}\left(\frac{a-kl}{2}\xi + \frac{a+kl}{2}\right)g_m\left(\frac{a-kl}{2}\xi + \frac{a+kl}{2}\right)
\end{equation}
This yields:

\begin{equation}
    M_{km} = \left(\frac{kl}{2}\right)^{1+\alpha} \int\limits_{-1}^{1} \tilde g_{1m}(\xi) (1+\xi)^{\alpha}d\xi + \left(\frac{a-kl}{2}\right)^{1 + \alpha}\int\limits_{-1}^{1} \tilde g_{2m}(\xi) (1-\xi)^{\alpha}d\xi
\end{equation}
Performing a Gauss-Jacobi integration scheme, we get:

\begin{equation}
    M_{km} \approx \left(\frac{kl}{2}\right)^{1+\alpha} \sum_{i}\tilde g_{1m}(\xi_i) w_i^{(\alpha,0)}+ \left(\frac{a-kl}{2}\right)^{1 + \alpha} \sum_i\tilde g_{1m}(\xi_i) w_i^{(0,\alpha)}
\end{equation}
where $w_i^{(\alpha,0)}$ means that we use the Gauss-Jacobi integration nodes and weights for $\alpha=-0.5 \, , \beta=0$ and so on.

\subsection*{Special case k=n}

For the case that $k=n$, we follow the same procedure as above:

\begin{equation}
  M_{nn} = \int\limits_{0}^{a} \phi_{en}(x) f_{n}(x) (a-x)^{\alpha} (a-x)^{-\alpha}dx
\end{equation}
We set $f_{n}(x) (a-x)^{-\alpha} = g_n(x)$, so:

\begin{align*}
    M_{nn} &= \int\limits_{0}^{a} \phi_{en}(x) g_n(x)(a-x)^{\alpha}dx \\
    &= \frac{a}{2} \int\limits_{-1}^{1} \phi_{en} \left(\frac{a}{2}\xi + \frac{a}{2}\right) g_n \left(\frac{a}{2}\xi + \frac{a}{2}\right) \left(\frac{a}{2}\right)^{\alpha} (1-\xi)^{\alpha}d\xi \\
    &=\left(\frac{a}{2}\right)^{1+\alpha} \int\limits_{-1}^{1}\phi_{en} \left(\frac{a}{2}\xi + \frac{a}{2}\right) g_n \left(\frac{a}{2}\xi + \frac{a}{2}\right) (1-\xi)^{\alpha}d\xi
\end{align*}
If we substitute:

\begin{equation}
    \tilde g_n(\xi) = \phi_{en}\left(\frac{a}{2}\xi + \frac{a}{2}\right)g_n\left(\frac{a}{2}\xi + \frac{a}{2}\right)
\end{equation}
we can approximate the integral above as:

\begin{equation}
    M_{nn} \approx \left(\frac{a}{2}\right)^{1+\alpha} \sum_i \tilde g(\xi) w_i^{(\alpha, 0)}
\end{equation}


\end{document}
